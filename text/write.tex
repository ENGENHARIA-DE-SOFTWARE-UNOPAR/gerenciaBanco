
\section{Introdução}


\noindent \begin{minipage}[c]{0.6\textwidth}
  \vspace {1cm}
  \begin{description}
    \item [Banana] Exemplo de mini página com figura e seus respectivos rotulos, para que sejam referenciados ao decorrer do texto.
    \item [Maça] Veja que a Figura \ref{fig:place}, está reservando um espaço para adição de figuras, e o mesmo já esta referenciando seu autor e sua nomeclatura com o indice automatico.
  \end{description}

\end{minipage}
\begin{minipage}[c]{0.4\textwidth}

  \includegraphics[width=\textwidth]{figure/placeholder.jpg}
  	\label{fig:place}
    \captionof{figure}{Placeholder, \cite{linux:2023}}
    %\captionof*{figure}{Fonte: \citeonline{linux:2023}}
\end{minipage}


\section{Métodos}


\section{Resultados}






\subsection{Código externo no main.c}

\lstinputlisting[language=java]{GerenciaBanco.java} %Busca os codigos na pasta /cod






\begin{equation}
S = \left\{
\begin{aligned}
  a + b     &= 4\\
  a \cdot b &= 4
\end{aligned}
\right.
$$ $$% Usado para pular linha [não recomendado]
\sum_{n<k,\;\text{$n$ odd}} nE_n
 \label{1}
\end{equation}


\section{Conclusões}


\begin{enumerate}[label=\Roman{*}, ref=(\roman{*})]
  \item fsfsdf
  \item kugfhiuh
\end{enumerate}

\begin{asparaenum}
\item Anterior ... \cite{ninguem2022curioso}
\item Próximo ... \label{pl1}
\end{asparaenum}









  %$X \xLongleftarrow[\text{NATAN}]{\text{OGLIARI}} Y $ %COM TEXTO
	% $\uparrow$ %Seta para Cima
	%$\overleftarrow{NATAN}$
