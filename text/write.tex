
\section{Introdução}


\noindent \begin{minipage}[c]{0.6\textwidth}
  \vspace {1cm}
  \par Esta presente aula pratica tem por fim a aplicação dos paradigmas da linguagem orientada a objetos com a linguagem de programação Java\ref{fig:log_java}.

\end{minipage}
\begin{minipage}[c]{0.4\textwidth}

  \includegraphics[width=\textwidth]{figure/log_java.jpg}
  	\label{fig:log_java}
    \captionof{figure}{Logo Java, \cite{logJava}}
    %\captionof*{figure}{Fonte: \citeonline{linux:2023}}
\end{minipage}


\section{Métodos}


\section{Resultados}






\subsection{Operações das classes}

\begin{lstlisting}[language=Java, caption=consultaPilas, label=consultaPilas]
    /**
     * @param nome da conta a ser consultado
     * @return O saldo da conta
     */
     public double consultaPilas(String nome){

         return this.saldo;
     }

\end{lstlisting}

\begin{lstlisting}[language=Java, caption=depositoPilas, label=depositaPilas]
    /**
   * @param nome da conta a ser depositado
   * @return 1 se deu certo e 0 se ocorreu um erro
   */
   public void depositaPila(double pilas, String nome){

       this.saldo += pilas;
   }
\end{lstlisting}


%\lstinputlisting[language=Java]{GerenciaBanco.java} %Busca os codigos na pasta /cod






\begin{equation}
S = \left\{
\begin{aligned}
  a + b     &= 4\\
  a \cdot b &= 4
\end{aligned}
\right.
$$ $$% Usado para pular linha [não recomendado]
\sum_{n<k,\;\text{$n$ odd}} nE_n
 \label{1}
\end{equation}


\section{Conclusões}


\begin{enumerate}[label=\Roman{*}, ref=(\roman{*})]
  \item fsfsdf
  \item kugfhiuh
\end{enumerate}

\begin{asparaenum}
\item Anterior ... \cite{ninguem2022curioso}
\item Próximo ... \label{pl1}
\end{asparaenum}









  %$X \xLongleftarrow[\text{NATAN}]{\text{OGLIARI}} Y $ %COM TEXTO
	% $\uparrow$ %Seta para Cima
	%$\overleftarrow{NATAN}$
